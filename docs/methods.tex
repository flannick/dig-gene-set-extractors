\documentclass[11pt]{article}

\usepackage{amsmath, amssymb}
\usepackage{geometry}
\usepackage{hyperref}
\usepackage{booktabs}
\usepackage{enumitem}

\geometry{margin=1in}

\title{Supplementary Note: Extracting Gene Programs (100--500 genes) from ATAC-seq}
\author{}
\date{}

\begin{document}
\maketitle

\section*{Motivation and scope}
The goal of these converters is not merely to output a gene-weight vector over all genes, but to extract compact gene sets (on the order of 100--500 genes) that represent an interpretable \emph{gene program} or \emph{mechanism} for downstream enrichment.
ATAC-seq measures chromatin accessibility at regulatory loci. Mapping ATAC signal to genes is therefore a model-based projection from locus space to gene space. In addition, extracting a compact program requires an explicit \emph{selection} step to identify the most informative genes for a given dataset, condition, or cell group.

This note provides:
\begin{itemize}[leftmargin=2em]
\item a shared mathematical framework for ATAC-to-gene scoring,
\item a family of peak-to-gene linkage models (promoter, distance, priors, co-accessibility),
\item and, critically, a set of \emph{program extraction operators} that convert gene scores into a 100--500 gene set with optional within-set weights.
\end{itemize}

\section{Notation}
\begin{itemize}[leftmargin=2em]
\item Peaks (regions) indexed by $p \in \{1,\dots,P\}$ with coordinates $(c_p,s_p,e_p)$.
\item Genes indexed by $g \in \{1,\dots,G\}$ with chromosome $c_g$, strand $\sigma_g \in \{+,-\}$, TSS position $t_g$, and promoter interval $\mathcal{P}_g$.
\item Peak-level observed statistic $x_p$ (counts, coverage, or differential statistic).
\item Peak weight $\alpha_p$ derived from $x_p$ via a transform $\phi(\cdot)$.
\item Peak-to-gene linkage weight $L_{pg} \ge 0$.
\item Raw gene score $s_g$ and (optional) reported gene weight $w_g$.
\end{itemize}

\section{Unified model: peak weights and peak-to-gene links}

\subsection{Peak weights}
Define a peak weight transform $\phi$:
\[
\alpha_p = \phi(x_p).
\]
Common transforms:
\begin{align*}
\phi_{\mathrm{signed}}(x) &= x, \\
\phi_{\mathrm{abs}}(x) &= |x|, \\
\phi_{\mathrm{pos}}(x) &= \max(x, 0), \\
\phi_{\mathrm{neg}}(x) &= \max(-x, 0).
\end{align*}
For program extraction, \texttt{pos} and \texttt{neg} are often preferable when $x_p$ is differential, because they yield direction-specific programs (opening vs closing chromatin).

\subsection{Gene scores as a linear projection}
We compute a raw gene score:
\begin{equation}
s_g \;=\; \sum_{p=1}^P \alpha_p \, L_{pg}.
\label{eq:gene_score}
\end{equation}
Equation \eqref{eq:gene_score} is a pragmatic surrogate for cis-regulatory influence: peaks with large $\alpha_p$ contribute more, and $L_{pg}$ encodes how peak $p$ is attributed to gene $g$.

\section{From gene scores to gene programs: selection and calibration}

\subsection{Why selection is necessary}
If we output $s_g$ for all genes and then normalize to sum to 1 (L1), the resulting weights can become small and nearly uniform when many genes receive some mass. This is expected mathematically but undesirable for extracting a compact, interpretable program.
Therefore, we explicitly define a \emph{selection operator} that turns scores $\{s_g\}$ into a gene set of target size 100--500.

\subsection{Selection operator}
Let $\mathbf{s} = (s_1,\dots,s_G)$ denote raw scores.
A selection operator $\mathcal{S}$ outputs an index set of genes $\widehat{\mathcal{G}} \subseteq \{1,\dots,G\}$ and optionally within-set weights.

\subsubsection*{Operator S1: top-$K$ selection (recommended default)}
\[
\widehat{\mathcal{G}}_K \;=\; \{ g : s_g \text{ is among the } K \text{ largest scores} \}.
\]
This is equivalent to solving:
\[
\widehat{\mathcal{G}}_K = \arg\max_{\mathcal{G}:|\mathcal{G}|=K} \sum_{g \in \mathcal{G}} s_g.
\]
Within the selected set, define optional weights:
\begin{equation}
w_g =
\begin{cases}
\dfrac{s_g}{\sum_{h \in \widehat{\mathcal{G}}_K} s_h}, & g \in \widehat{\mathcal{G}}_K \\
0, & \text{otherwise.}
\end{cases}
\label{eq:within_set_l1}
\end{equation}
This yields a compact program of exactly $K$ genes (e.g., $K \in [100,500]$) and interpretable relative weights within the program.

\subsubsection*{Operator S2: quantile selection}
Let $q_\tau(\mathbf{s})$ denote the $(1-\tau)$ quantile of scores, for $\tau \in (0,1)$:
\[
\widehat{\mathcal{G}}_\tau \;=\; \{ g : s_g \ge q_\tau(\mathbf{s}) \}.
\]
This selects a fraction of genes rather than a fixed number. It is useful when $G$ varies across annotations, but may not guarantee 100--500 genes without tuning $\tau$.

\subsubsection*{Operator S3: threshold selection}
\[
\widehat{\mathcal{G}}_t \;=\; \{ g : s_g \ge t \}.
\]
This is useful when $s_g$ has a stable absolute meaning (rare for ATAC without calibration). In practice, $t$ is usually chosen to match a desired set size.

\subsubsection*{Operator S4: softmax calibration to a target effective size (optional)}
If one wants a simplex-valued weight vector but not near-uniformity, define a temperature-scaled softmax:
\[
w_g(T) = \frac{\exp(s_g/T)}{\sum_h \exp(s_h/T)}.
\]
Define the effective number of genes:
\[
N_{\mathrm{eff}}(T) = \frac{1}{\sum_g w_g(T)^2}.
\]
Choose $T$ (via binary search) so that $N_{\mathrm{eff}}(T)$ lies in a target range (e.g., 200--400).
One can then take the top-$K$ genes under $w_g(T)$ for reporting.
This approach is optional; top-$K$ on $s_g$ is simpler and usually sufficient.

\subsection{What the resulting program represents}
Programs can represent different biological notions depending on how $\alpha_p$ is constructed:
\begin{itemize}[leftmargin=2em]
\item \textbf{Activity program:} genes with the highest overall promoter or linked accessibility in a sample or group.
\item \textbf{Specificity program (recommended for scATAC clusters):} genes most specifically accessible in a group versus a background.
\item \textbf{Directional program:} opening versus closing programs using positive/negative transforms on differential peak statistics.
\end{itemize}

\section{Peak-to-gene linkage models $L_{pg}$}
We summarize linkage models used in Eq.\ \eqref{eq:gene_score}. All models can be paired with any selection operator.

\subsection{Model A: promoter overlap (high specificity, misses distal enhancers)}
\paragraph{Biology.} Promoter accessibility is a direct indicator of local transcriptional competence and is least ambiguous to map.
\paragraph{Definition.}
\[
L_{pg} = \mathbf{1}\{ c_p=c_g \wedge [s_p,e_p) \cap \mathcal{P}_g \neq \emptyset \}.
\]
\paragraph{Heuristics.} Typical promoter window defaults $(u,d)=(2000,500)$ bp; expose as parameters. Consider union across transcripts or canonical TSS only; record choice.

\subsection{Model B: nearest TSS assignment (simple distal inclusion)}
\paragraph{Biology.} Many enhancers act on nearby genes; nearest-gene is a coarse but common heuristic.
\paragraph{Definition.} Let $m_p=(s_p+e_p)/2$ and $d(p,g)=|m_p-t_g|$. If $d(p,g)\le D$:
\[
g^*(p)=\arg\min_g d(p,g), \quad L_{pg}=\mathbf{1}\{g=g^*(p)\}.
\]
\paragraph{Heuristics.} Use per-chromosome sorted TSS arrays for $O(\log G)$ nearest lookup; cap by max distance $D$.

\subsection{Model C: distance-decay soft assignment (recommended distal-aware heuristic)}
\paragraph{Biology.} Regulatory influence and contact probability tend to decrease with genomic distance on average.
\paragraph{Definition.} For distance $d(p,g)$, max distance $D$, and decay length $\lambda$:
\[
A_{pg} =
\begin{cases}
\exp(-d(p,g)/\lambda), & d(p,g)\le D \\
0, & \text{otherwise}
\end{cases}
\]
Either use $L_{pg}=A_{pg}$ (unnormalized) or normalize per peak:
\[
L_{pg} = \frac{A_{pg}}{\sum_h A_{ph}+\epsilon}.
\]
\paragraph{Heuristics.} Defaults such as $D=500$ kb, $\lambda=50$ kb; cap links per peak to top $K_{\max}$ genes by $A_{pg}$.

\subsection{Model D: gene body activity (gene activity / gene score)}
\paragraph{Biology.} Aggregated accessibility across promoter and gene body can correlate with transcriptional activity and provides robustness in sparse scATAC.
\paragraph{Definition.} Define a gene locus interval $\mathcal{G}_g$ (gene body plus upstream extension) and sum accessibility over overlapping peaks:
\[
s_g = \sum_{p: [s_p,e_p) \cap \mathcal{G}_g \neq \emptyset} \alpha_p.
\]
\paragraph{Heuristics.} Mitigate gene-length bias by using promoter-only, length normalization, or restricting to standardized windows.

\subsection{Model E: co-accessibility linkage (data-driven enhancer to promoter)}
\paragraph{Biology.} Peaks that co-vary across cells can reflect regulatory coupling and enhancer-promoter communication.
\paragraph{Definition.} Estimate co-accessibility $C_{p,q(g)}$ between distal peak $p$ and promoter peak(s) $q(g)$ for gene $g$:
\[
L_{pg} = \max(C_{p,q(g)},0)\cdot \mathbf{1}\{d(p,g)\le D\}.
\]
\paragraph{Heuristics.} Requires many cells; restrict candidate pairs within $D$; threshold edges; optionally combine with distance-decay.

\subsection{Model F: external enhancer-gene priors}
\paragraph{Biology.} External maps integrate multi-assay evidence and can improve specificity.
\paragraph{Definition.} Overlap peaks to external elements $e$ with link matrix $M_{eg}$:
\[
L_{pg} = \sum_e \mathbf{1}\{p \cap e \neq \emptyset\} M_{eg}.
\]
\paragraph{Heuristics.} Choose a default map per genome build; record version; handle biosample mismatch via closest match or aggregate union.

\subsection{Model G: Activity-by-Contact style linkage}
\paragraph{Biology.} Enhancer effect is proportional to activity and contact with a promoter.
\paragraph{Definition.} With activity $A_p$ and contact kernel $K_{pg}$:
\[
S_{pg} = A_p K_{pg}, \qquad L_{pg} = \frac{S_{pg}}{\sum_h S_{ph}+\epsilon}.
\]
\paragraph{Heuristics.} If no contact map, approximate $K_{pg}$ by a distance kernel (power law or exponential) and record that approximation.

\subsection{Model H: TF-centric programs (motif activity to gene targets)}
\paragraph{Biology.} ATAC is sensitive to TF-driven regulatory programs; infer TF activity and project to target genes.
\paragraph{Definition.} With TF activities $a_k$ and TF-to-gene prior $T_{kg}$:
\[
s_g = \sum_{k=1}^K a_k T_{kg}.
\]
\paragraph{Heuristics.} Depends on a TF-target prior; useful as a complementary view.

\section{Bulk ATAC-seq programs}
Bulk ATAC programs are most interpretable when driven by contrasts (differential accessibility). Single-sample programs are possible but often highlight broadly accessible promoters.

\subsection{Bulk model B1: single-sample activity program}
\paragraph{Peak weights.} Use nonnegative peak signal, e.g., normalized coverage or narrowPeak signalValue: $\alpha_p = x_p$.
\paragraph{Link.} Model A (promoter) or C (distance-decay).
\paragraph{Program extraction.} Use top-$K$ selection (S1) on $s_g$ to get an activity program.
\paragraph{Heuristic.} Choose $K=200$ by default; optionally exclude mitochondrial or blacklisted genes.

\subsection{Bulk model B2: differential opening/closing programs (recommended)}
\paragraph{Peak weights.} Let $x_p$ be a signed differential statistic (log fold-change or Wald z). Define:
\[
\alpha^{\mathrm{open}}_p = \max(x_p,0), \qquad \alpha^{\mathrm{close}}_p = \max(-x_p,0).
\]
\paragraph{Link.} Prefer Model C (distance-decay) to capture distal regulation; Model A for promoter-specific effects.
\paragraph{Program extraction.} Produce two programs via S1:
\[
\widehat{\mathcal{G}}^{\mathrm{open}}_K = \mathcal{S}_K(\mathbf{s}^{\mathrm{open}}), \quad
\widehat{\mathcal{G}}^{\mathrm{close}}_K = \mathcal{S}_K(\mathbf{s}^{\mathrm{close}}).
\]
\paragraph{Heuristic.} $K=200$ each direction; record direction in metadata.

\subsection{Bulk model B3: differential with external priors}
Use Model F or G for $L_{pg}$ if available; proceed as in B2 for program extraction.

\section{Single-cell ATAC-seq programs}
For scATAC, group-level (cluster-level) programs are typically more stable for enrichment than per-cell gene scores.

\subsection{Peak summaries for groups}
Let $X_{pc}$ be peak-by-cell counts (or binary).
For a group $\mathcal{C}$ define:
\begin{itemize}[leftmargin=2em]
\item sum counts: $\bar{x}_p = \sum_{c\in\mathcal{C}} X_{pc}$
\item mean counts: $\bar{x}_p = \frac{1}{|\mathcal{C}|}\sum_{c\in\mathcal{C}} X_{pc}$
\item fraction nonzero: $\bar{x}_p = \frac{1}{|\mathcal{C}|}\sum_{c\in\mathcal{C}} \mathbf{1}\{X_{pc}>0\}$
\end{itemize}
Then $\alpha_p = \bar{x}_p$ defines an absolute activity program.

\subsection{scATAC model S1: absolute group activity program}
\paragraph{Biology.} Captures genes with high promoter (or linked) accessibility in the group.
\paragraph{Model.} Compute $\alpha_p=\bar{x}_p$, link with Model A or C, compute $s_g$ and select top-$K$ (S1).
\paragraph{Caveat.} Can be dominated by broadly accessible promoters; may not yield cluster-specific programs.

\subsection{scATAC model S2: group-specific differential program (recommended default for clusters)}
\paragraph{Biology.} Captures genes whose regulatory regions are more accessible in the group than in a background (all other cells), aligning with marker programs and mechanisms.
\paragraph{Definition.} Let $\bar{x}^{(1)}_p$ be group summary and $\bar{x}^{(0)}_p$ be background summary. Define a contrast score:
\begin{equation}
x_p = \log_2\left(\frac{\bar{x}^{(1)}_p + a}{\bar{x}^{(0)}_p + a}\right),
\label{eq:sc_logfc}
\end{equation}
where $a>0$ is a pseudocount (chosen based on summary type). Then use:
\[
\alpha_p = \max(x_p,0)
\]
to obtain a group-specific opening program, link to genes, compute $s_g$, and select top-$K$.
\paragraph{Heuristics.} For fraction nonzero, use small $a$ (e.g., $10^{-3}$); for counts, use $a=1$. Optionally also emit closing programs via $\max(-x_p,0)$.

\subsection{scATAC model S3: gene activity window}
Compute gene activity $s_g$ over promoter+gene body (Model D), then select top-$K$ for a group or differential contrast (apply Eq.\ \eqref{eq:sc_logfc} to gene activity scores).

\subsection{scATAC model S4: co-accessibility-enhanced programs}
Estimate co-accessibility links (Model E), compute $s_g$ and select top-$K$.
This yields programs that better reflect enhancer-promoter wiring, at higher computational cost.

\subsection{scATAC model S5: latent program extraction (optional extension)}
\paragraph{Biology.} Cell states can be described by multiple overlapping programs (topics). A single top-$K$ list per cluster may miss subprograms.
\paragraph{Model.} Factorize the peak-by-cell matrix using a nonnegative factorization:
\[
X \approx W H,
\]
where $W$ is peak-by-topic and $H$ is topic-by-cell. For topic $k$, define topic peak weights $\alpha^{(k)}_p = W_{pk}$, map to genes via $L_{pg}$, and select top-$K$ genes as the topic program.
\paragraph{Heuristic.} Use a small number of topics (e.g., 10--50), then treat each topic as a program.

\section{Recommended defaults for program-sized outputs}
For the objective of extracting 100--500 gene programs suitable for enrichment:
\begin{itemize}[leftmargin=2em]
\item \textbf{Bulk ATAC:} differential opening and closing programs (B2) with distance-decay links (C), select top-$K$ with $K=200$ per direction.
\item \textbf{scATAC clusters:} group-specific differential programs (S2) using log2 fold-change (Eq.\ \ref{eq:sc_logfc}), promoter overlap (A) as baseline and distance-decay (C) as distal-aware alternative, select top-$K$ with $K=200$.
\item \textbf{Output:} report raw score $s_g$ and a within-program normalized weight (Eq.\ \ref{eq:within_set_l1}) for interpretability; avoid global L1 normalization over all genes as the primary output when the goal is a compact program.
\end{itemize}

\section{Implementation notes (practical)}
\begin{itemize}[leftmargin=2em]
\item Always record: genome build, GTF source/version, link method and parameters, peak statistic source, contrast definition, and selection method ($K$ or quantile).
\item Provide diagnostics: number of peaks assigned to any gene; low assignment suggests genome build or chromosome naming mismatch.
\item Ensure deterministic ranking: sort by decreasing score, tie-break by gene_id.
\end{itemize}

\end{document}
